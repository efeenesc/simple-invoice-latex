\documentclass[letterpaper]{article}
\usepackage{latexsym}
\usepackage{titlesec}
\usepackage{fontawesome5}
\usepackage[usenames,dvipsnames]{color}
\usepackage{verbatim}
\usepackage{enumitem}
\usepackage[hidelinks]{hyperref}
\usepackage{fancyhdr}
\usepackage[english]{babel}
\usepackage{tabularx}
\usepackage{ragged2e}
\usepackage{etoolbox}
\usepackage{tikz}
\usepackage{fontspec}
\usepackage{lastpage} % Total page count
\usepackage[margin=0.5in, top=0.5in, bottom=0.7in, footnotesep=0cm]{geometry} % Page margins
\usepackage{parskip}
\usepackage{colortbl}
\usepackage[symbol*]{footmisc}
\setfnsymbol{wiley}

\definecolor{offwhite}{HTML}{9C9C9C}

\setmainfont{Helvetica Neue}  % or another system font

\hypersetup{
    urlcolor=blue
}

\newcommand{\LARGER}{\fontsize{24}{24}\selectfont}

\begin{document}
	
\thispagestyle{empty}


% Header, for company, invoice info

\hfill{\includegraphics[height=1.98cm]{logo.png}}
\vspace{-1.98cm}

\textbf{\LARGER{Invoice}}
\vspace{20pt}

\renewcommand{\arraystretch}{1.2}

\begin{table}[h]
\begin{tabular}{@{}l l}
\textbf{Invoice number} & <<INV-NUMBER>> \\
\textbf{Date of issue}  & <<DATE-ISSUE>> \\ % always the 23rd of this month
\textbf{Delivery period}& <<DATE-DELIVERY>> \\ % always the last day of this month
\textbf{Date due}       & <<DATE-DUE>> % 1 week after the 27th of this month (e.g. Aug 3 if month is July)
\end{tabular}
\end{table}

\begin{table}[h]
\begin{tabularx}{\textwidth}{@{}l l l}
\textbf{<<ISSUER-NAME>>} && \textbf{Bill to}        \\
<<ISSUER-ADDRESS-1>>    && <<BILL-TO-NAME>>       \\
<<ISSUER-ADDRESS-2>>    && <<BILL-TO-ADDRESS-1>>  \\
<<ISSUER-ADDRESS-3>>    && <<BILL-TO-ADDRESS-2>>  \\
<<ISSUER-PHONE>>        && <<BILL-TO-LINE-1>>     \\
<<ISSUER-EMAIL>>        && <<BILL-TO-LINE-2>>
\end{tabularx}
\end{table}

\renewcommand{\arraystretch}{1.5}

\normalsize{<<PAYMENT-NOTES>>}

\begin{table}[h]
\setlength{\tabcolsep}{12pt} % Default is usually 6pt
\begin{tabularx}{\textwidth}{@{}X r r r r@{}}
   \textbf{Description} & \textbf{Qty} & \textbf{Unit price} & \textbf{Discount} & \textbf{Amount} \\[0.2em]
   \hline
   <<LINE-ITEMS>>
\end{tabularx}
\end{table}

\vspace{10pt}

\hfill
\begin{tabularx}{0.5\textwidth}{l >{\raggedleft\arraybackslash}X}
    Subtotal                    & <<SUBTOTAL>>              \\
    \arrayrulecolor{offwhite}\hline
    Discount                    & <<DISCOUNT-TOTAL>>        \\
    \hline
    Tax                         & <<TAX-PERCENT>>           \\
    \hline
    Total                       & <<TOTAL>>                 \\
    \hline
    \large\textbf{Amount due}   & \large\textbf{<<TOTAL>>}
\end{tabularx}

\vspace{2em}
\vfill

\LARGE\textbf{Payment details}\vspace{0.2em}

\normalsize{Payments should be made in \textbf{USD} via SWIFT to the following bank account:}

\begin{tabular}{@{}l l}
\textbf{Bank name}              & <<BANK-NAME>> \\
\textbf{Bank address}           & <<BANK-ADDRESS>> \\
\textbf{IBAN}                   & <<BANK-ACCOUNT-IBAN>> \\ % always the 23rd of this month
\textbf{Account holder name}    & <<BANK-ACCOUNT-NAME>> \\
\textbf{Account number}         & <<BANK-ACCOUNT-NUMBER>> \\
\textbf{SWIFT code}             & <<BANK-SWIFT-CODE>>
\end{tabular}

\vspace{10pt}
\end{document}